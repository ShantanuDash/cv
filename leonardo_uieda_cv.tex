%%%%%%%%%%%%%%%%%%%%%%%%%%%%%%%%%%%%%%%%%%%%%%%%%%%%%%%%%%%%%%%%%%%%%%%%%%%%%%%
% A clean template for an academic CV
%
% Uses tabularx to create two column entries (date and job/edu/citation).
% Defines commands to make adding entries simpler.
%
%%%%%%%%%%%%%%%%%%%%%%%%%%%%%%%%%%%%%%%%%%%%%%%%%%%%%%%%%%%%%%%%%%%%%%%%%%%%%%%

\documentclass[11pt, a4paper]{article}

% Useful aliases
\newcommand{\UERJ}{Universidade do Estado do Rio de Janeiro}
\newcommand{\UHM}{University of Hawai`i at M\={a}noa}
\newcommand{\SOEST}{School of Ocean and Earth Science and Technology}
\newcommand{\UHEARTH}{Department of Earth Sciences}
\newcommand{\LIVEARTH}{Department of Earth, Ocean and Ecological Sciences}
\newcommand{\LIVENV}{School of Environmental Sciences}
\newcommand{\LIV}{University of Liverpool}

% Identifying information
\newcommand{\Title}{Curriculum Vit\ae}
\newcommand{\FirstName}{Leonardo}
\newcommand{\LastName}{Uieda}
\newcommand{\Initials}{L}
\newcommand{\MyName}{\FirstName\ \LastName}
\newcommand{\Me}{\textbf{\LastName, \Initials}}  % For citations
\newcommand{\Email}{uieda@liverpool.ac.uk}
\newcommand{\Website}{leouieda.com}
\newcommand{\Lab}{compgeolab.org}
\newcommand{\ORCID}{0000-0001-6123-9515}
\newcommand{\Affiliation}{\LIVEARTH \\ \LIVENV \\ \LIV}
\newcommand{\Address}{
  Jane Herdman Building, 4 Brownlow Street \\ L69 3GP, Liverpool, United Kingdom
}

% Names for citing coauthors
\newcommand{\Val}{Barbosa, VCF}
\newcommand{\Bi}{Oliveira Jr, VC}
\newcommand{\Paul}{Wessel, P}
\newcommand{\Joaquim}{Luis, J}
\newcommand{\Remko}{Scharroo, R}
\newcommand{\Florian}{Wobbe, F}
\newcommand{\Walter}{Smith, WHF}
\newcommand{\Dongdong}{Tian, D}
\newcommand{\Bridget}{Smith-Konter, B}
\newcommand{\Eric}{Xu, X}
\newcommand{\David}{Sandwell, DT}
\newcommand{\Carla}{Braitenberg, C}
\newcommand{\Naomi}{Ussami, N}
\newcommand{\Manoel}{D'Agrella-Filho, MS}
\newcommand{\JB}{Silva, JBC}
\newcommand{\Dai}{Sales, DP}
\newcommand{\Figura}{Melo, FF}
\newcommand{\Dio}{Carlos, DU}
\newcommand{\BragaVale}{Braga, MA}
\newcommand{\YLi}{Li, Y}
\newcommand{\Angeli}{Angeli, G}
\newcommand{\Peres}{Peres, G}
\newcommand{\Everton}{Bomfim, EP}
\newcommand{\Eder}{Molina, E}
\newcommand{\Gomes}{Gomes, AAS}
\newcommand{\Santiago}{Soler, SR}
\newcommand{\Agustina}{Pesce, A}
\newcommand{\Gimenez}{Gimenez, ME}
\newcommand{\Kristoffer}{Hallam, KAT}
\newcommand{\Guangdong}{Zhao, G}
\newcommand{\Bo}{Chen, B}
\newcommand{\JLiu}{Liu, J}
\newcommand{\LChen}{Chen, L}
\newcommand{\RGuo}{Guo, R}
\newcommand{\MKaban}{Kaban, MK}
\newcommand{\Lindsey}{Heagy, LJ}
\newcommand{\Lion}{Krischer, L}
\newcommand{\Rene}{Gassmoeller, R}
\newcommand{\Bane}{Sullivan, CB}
\newcommand{\Jens}{Klump, JF}
\newcommand{\LBarba}{Barba, LA}
\newcommand{\JBazan}{Bazan, J}
\newcommand{\JBrown}{Brown, J}
\newcommand{\RGuimera}{Guimera, RV}
\newcommand{\MGymrek}{Gymrek, M}
\newcommand{\AHanna}{Alex Hanna}
\newcommand{\KHuff}{Huff, KD}
\newcommand{\DKatz}{Katz, DS}
\newcommand{\CMadan}{Madan, CR}
\newcommand{\KMoerman}{Moerman, KM}
\newcommand{\KNiemeyer}{Niemeyer, KE}
\newcommand{\JPoulson}{Poulson, JL}
\newcommand{\PPrins}{Prins, P}
\newcommand{\KRam}{Ram, K}
\newcommand{\ARokem}{Rokem, A}
\newcommand{\Arfon}{Smith, AM}
\newcommand{\GThiruvathukal}{Thiruvathukal, GK}
\newcommand{\KThyng}{Thyng, KM}
\newcommand{\BWilson}{Wilson, BE}
\newcommand{\Yehudi}{Yehudi, Y}
\newcommand{\Remi}{Rampin, R}
\newcommand{\Hugo}{van Kemenade, H}
\newcommand{\MattTurk}{Turk, M}
\newcommand{\Shapero}{Shapero, D}
\newcommand{\Anderson}{Banihirwe, A}
\newcommand{\Leeman}{Leeman, J}
\newcommand{\JEbbing}{Ebbing, J}
\newcommand{\AGuy}{Guy, A}
\newcommand{\JFarquharson}{Farquharson, J}
\newcommand{\AKushnir}{Kushnir, A}
\newcommand{\FWadsworth}{Wadsworth, F}


% Template configuration
%%%%%%%%%%%%%%%%%%%%%%%%%%%%%%%%%%%%%%%%%%%%%%%%%%%%%%%%%%%%%%%%%%%%%%%%%%%%%%%

% Disable hyphenation
\usepackage[none]{hyphenat}

% Control the font size
\usepackage{anyfontsize}

% Template variables for styling
\newcommand{\TablePad}{\vspace{-0.4cm}}
\newcommand{\SoftwareTitle}[1]{{\bfseries #1}}
\newcommand{\TableTitle}[1]{{\fontsize{12pt}{0}\selectfont \itshape #1}}
\newcommand{\Invited}{\textbf{[Invited]}}

% For fancy and multipage tables
\usepackage{tabularx}
\usepackage{ltablex}

% Define a new environment to place all CV entries in a 2-column table.
% Left column are the dates, right column the entries.
\usepackage{environ}
\NewEnviron{EntriesTable}{
\TablePad
\begin{tabularx}{\textwidth}{@{}p{0.1\textwidth}@{\hspace{0.02\textwidth}}p{0.88\textwidth}@{}}
  \BODY
\end{tabularx}
}

% Macros to add links
\newcommand{\DOI}[1]{doi:\href{https://doi.org/#1}{#1}}
\newcommand{\Preprint}[1]{Preprint: \href{https://doi.org/#1}{doi.org/#1}}
\newcommand{\Youtube}[1]{Recording: \href{https://youtu.be/#1}{youtu.be/#1}}
\newcommand{\GitHub}[1]{GitHub: \href{https://github.com/#1}{#1}}

% Tags to mark publications
%\newcommand{\OA}{{\bfseries [open access]}}
\newcommand{\OA}{}

% Macros to set the year and duration on the left column
\newcommand{\Duration}[2]{\fontsize{10pt}{0}\selectfont #1--#2}
\newcommand{\Year}[1]{\fontsize{10pt}{0}\selectfont #1}
\newcommand{\Ongoing}{}
%\newcommand{\Ongoing}{$\ast$}
\newcommand{\Future}{future}
\newcommand{\Review}{in review}
\newcommand{\Accepted}{accepted}
\newcommand{\Appointment}[4]{\textbf{#1} \newline #2 \newline #3 \newline #4}

% Define command to insert month name and year as date
\usepackage{datetime}
\newdateformat{monthyear}{\monthname[\THEMONTH], \THEYEAR}

% Set the page margins
\usepackage[left=1in,right=1in,top=1in,bottom=1in]{geometry}

% No indentation
\setlength\parindent{0cm}

% Increase the line spacing
\renewcommand{\baselinestretch}{1.1}
% and the spacing between rows in tables
\renewcommand{\arraystretch}{1.5}

% Remove space between items in itemize and enumerate
\usepackage{enumitem}
\setlist{nosep}

% Use custom colors
\usepackage[usenames,dvipsnames]{xcolor}

% Set fonts. Requires compilation with xelatex
\usepackage{fontspec}  % required to make older xelatex compile with UTF8

% Configure the font style for sections
\usepackage{sectsty}
\sectionfont{\vspace{0.5cm}\bfseries\fontsize{12pt}{0}\selectfont\uppercase}
\subsectionfont{\vspace{0.2cm}\mdseries\fontsize{12pt}{0}\selectfont\uppercase}

% Set the spacing for sections
%\usepackage{titlesec}
%\titlespacing{\section}{0pt}{0cm}{0.3cm}
%\titlespacing{\subsection}{0pt}{0.3cm}{0.3cm}

% Disable number of sections. Use this instead of "section*" so that the sections still
% appear as PDF bookmarks. Otherwise, would have to add the table of contents entries
% manually.
\makeatletter
\renewcommand{\@seccntformat}[1]{}
\makeatother

% Set fancy headers
\usepackage{fancyhdr}
\pagestyle{fancy}
\fancyhf{}
\chead{
  \fontsize{10pt}{12pt}\selectfont
  \MyName
  \hspace{0.2cm} -- \hspace{0.2cm}
  \Title
  \hspace{0.2cm} -- \hspace{0.2cm}
  \monthyear\today
}
\rhead{}
\cfoot{\fontsize{10pt}{0}\selectfont \thepage}
\renewcommand{\headrulewidth}{0pt}

% Metadata for the PDF output and control of hyperlinks
\usepackage[colorlinks=true]{hyperref}
\hypersetup{
  pdftitle={\MyName\ - \Title},
  pdfauthor={\MyName},
  linkcolor=blue,
  citecolor=blue,
  filecolor=black,
  urlcolor=MidnightBlue
}
%%%%%%%%%%%%%%%%%%%%%%%%%%%%%%%%%%%%%%%%%%%%%%%%%%%%%%%%%%%%%%%%%%%%%%%%%%%%%%%


\begin{document}

% No header for the first page
\thispagestyle{empty}

%%%%%%%%%%%%%%%%%%%%%%%%%%%%%%%%%%%%%%%%%%%%%%%%%%%%%%%%%%%%%%%%%%%%%%%%%%%%%%%
% HEADER
{\fontsize{24pt}{0}\selectfont\MyName}\\[-0.1cm]
\rule{\textwidth}{0.2pt}
\begin{minipage}[t]{0.595\textwidth}
  \Affiliation
  \\
  \Address
\end{minipage}
\begin{minipage}[t]{0.405\textwidth}
  \begin{flushright}
  Last updated: \monthyear\today
  \\
    ORCID: \href{https://orcid.org/\ORCID}{\ORCID}
    \\
    email: \href{mailto:\Email}{\Email}
    \\
    Research group: \href{https://www.\Lab}{\Lab}
    \\
    Website: \href{https://www.\Website}{\Website}
  \end{flushright}
\end{minipage}

%%%%%%%%%%%%%%%%%%%%%%%%%%%%%%%%%%%%%%%%%%%%%%%%%%%%%%%%%%%%%%%%%%%%%%%%%%%%%%%
\section{Professional Appointments}

\begin{EntriesTable}
  \Duration{2019}{}  &
  \Appointment{Lecturer}{\LIVEARTH}{\LIVENV}{\LIV, UK}
  \\
  \Duration{2018}{}  &
  \Appointment{Affiliate Researcher}{\UHEARTH}{\SOEST}{\UHM, USA}
  \\
  \Duration{2017}{2018}  &
  \Appointment{Visiting Research Scholar}{\UHEARTH}{\SOEST}{\UHM, USA}
  \\
  \Duration{2014}{2018}  &
  \Appointment{Assistant Professor}{Departamento de Geologia Aplicada}{Faculdade de Geologia}{\UERJ, Brazil}
\end{EntriesTable}


%%%%%%%%%%%%%%%%%%%%%%%%%%%%%%%%%%%%%%%%%%%%%%%%%%%%%%%%%%%%%%%%%%%%%%%%%%%%%%%
\section{Education}

\begin{EntriesTable}
  \Duration{2011}{2016}  &
  \textbf{PhD in Geophysics}, Observatório Nacional, Brazil
  \\
  \Duration{2010}{2011}  &
  \textbf{MSc in Geophysics}, Observatório Nacional, Brazil
  \\
  \Duration{2008}{2009}  &
  \textbf{International Exchange} (1 year), York University, Canada
  \\
  \Duration{2004}{2009}  &
  \textbf{BSc in Geophysics}, Universidade de São Paulo, Brazil
\end{EntriesTable}


%%%%%%%%%%%%%%%%%%%%%%%%%%%%%%%%%%%%%%%%%%%%%%%%%%%%%%%%%%%%%%%%%%%%%%%%%%%%%%%
\section{Grants \& Fellowships}

\begin{EntriesTable}
  \Duration{2020}{2023}  &
  NSF-EAR: ``A Sustainable Plan for the Future of the Generic Mapping Tools''.
  PI: \Paul, \textbf{co-PI}: \Me.
  \textit{\UHM}.
  Award ID: \href{https://www.nsf.gov/awardsearch/showAward?AWD_ID=1948602}{1948602}.
  \\
  \Year{2020}  &
  Software Sustainability Institute Fellowship.
  \textit{\LIV}.
  Amount: £3000.
  More information:
  \href{https://www.leouieda.com/research/ssi2020.html}{leouieda.com/research/ssi2020.html}
  \\
  \Duration{2019}{2020}  &
  ESRG Research Support:
  ``Geophysical inversion of GRACE satellite time-lapse gravity''.
  Internal fund to support the research visit of PhD student Santiago R.
  Soler.
  \textit{\LIV}.
  Amount: £1000.
  \\
  \Duration{2018}{2020}  &
  NSF-EAR: ``The EarthScope/GMT Analysis and Visualization Toolbox''.
  PI: \Paul, \textbf{co-PI}: \Me, co-PI: \Bridget.
  \textit{\UHM}.
  Amount: \$174,975.
  Award ID: \href{https://www.nsf.gov/awardsearch/showAward?AWD_ID=1829371}{1829371}.
  \\
  \Duration{2014}{2018}  &
  QUALITEC/UERJ Grant for training a technician for the Laboratory of
  Exploration Geophysics - \UERJ
\end{EntriesTable}


%%%%%%%%%%%%%%%%%%%%%%%%%%%%%%%%%%%%%%%%%%%%%%%%%%%%%%%%%%%%%%%%%%%%%%%%%%%%%%%
\section{Awards \& Honors}

\begin{EntriesTable}
  \Year{2017}  &
  Brazilian Geophysical Society (SBGf) Award for \textbf{Best PhD Thesis}
  of 2015 -- 2017
  \\
  \Year{2016}  &
  \UERJ, Brazil, School of Geology
  \textbf{Teaching Award} given by the graduating class of 2016
  \\
  \Duration{2011}{2015}  &
  Brazilian Ministry of Education CAPES \textbf{PhD Research Scholarship}
  \\
  \Year{2011}  &
  SEG Near Surface Geophysics Section \textbf{Student Travel Grant} to
  present at the SEG Annual Meeting, San Antornio, TX, USA
  \\
  \Year{2011}  &
  EAGE \textbf{PACE Student Travel Grant} to present at the 73rd EAGE
  Conference \& Exhibition, Vienna, Austria
  \\
  \Duration{2010}{2011}  &
  Brazilian Ministry of Education CAPES \textbf{Masters Research Scholarship}
  \\
  \Year{2008}  &
  Brazilian Geophysical Society (SBGf) \textbf{Undergraduate Research
  Scholarship}
  \\
  \Year{2005}  &
  São Paulo Research Foundation (FAPESP) \textbf{Undergraduate Research
  Scholarship}
\end{EntriesTable}


%%%%%%%%%%%%%%%%%%%%%%%%%%%%%%%%%%%%%%%%%%%%%%%%%%%%%%%%%%%%%%%%%%%%%%%%%%%%%%%
\section{Publications}

\subsection{Preprints}

\begin{EntriesTable}
\Year{2019}  &
  \LBarba, \JBazan, \JBrown, \RGuimera, \MGymrek, \AHanna, \Lindsey, \KHuff, \DKatz,
  \CMadan, \KMoerman, \KNiemeyer, \JPoulson, \PPrins, \KRam, \ARokem, \Arfon,
  \GThiruvathukal, \KThyng, \Me, \BWilson, \Yehudi.
  Giving software its due through community-driven review and publication.
  \emph{OSF Preprints}.
  \DOI{10.31219/osf.io/f4vx6}
\end{EntriesTable}

\subsection{Peer-Reviewed}

\begin{EntriesTable}
%\Year{\Review}  &
\Year{2020}  &
  \Me, \Santiago, \Remi, \Hugo, \MattTurk, \Shapero, \Anderson, \Leeman.
  Pooch: A friend to fetch your data files.
  \emph{Journal of Open Source Software}.
  \DOI{10.21105/joss.01943}.
  \OA
  \newline
  \GitHub{fatiando/pooch}
  \\
\Year{2019}  &
  \Paul, \Joaquim, \Me, \Remko, \Florian, \Walter, \Dongdong.
  The Generic Mapping Tools, Version 6.
  \emph{Geochemistry, Geophysics, Geosystems}.
  \DOI{10.1029/2019GC008515}.
  \\
  ~ &
  \Santiago, \Agustina, \Gimenez, \Me.
  Gravitational field calculation in spherical coordinates using variable densities in
  depth.
  \emph{Geophysical Journal International}.
  \DOI{10.1093/gji/ggz277}.
  %\newline
  \Preprint{10.31223/osf.io/3548g}
  \newline
  \GitHub{pinga-lab/tesseroid-variable-density}
  \\
  ~ &
  \Guangdong, \Bo, \Me, \JLiu, \MKaban, \LChen, \RGuo.
  Efficient 3D large-scale forward-modeling and inversion of gravitational fields in
  spherical coordinates with application to lunar mascons.
  \emph{Journal of Geophysical Research: Solid Earth}.
  \DOI{10.1029/2019jb017691}.
  \Preprint{10.31223/osf.io/dzf9j}
  \\
\Year{2018}  &
  \Me. Verde: Processing and gridding spatial data using Green's functions.
  \emph{Journal of Open Source Software}.
  \DOI{10.21105/joss.00957}.
  \OA
  \newline
  \GitHub{fatiando/verde}
  \\
\Year{2017}  &
  \Me, \Val.
  Fast non-linear gravity inversion in spherical coordinates with application
  to the South American Moho,
  \emph{Geophysical Journal International},
  \DOI{10.1093/gji/ggw390}.
  \Preprint{10.31223/osf.io/9ba4m}
  \newline
  \GitHub{pinga-lab/paper-moho-inversion-tesseroids}
  \\
\Year{2016}  &
  \Me, \Val, \Carla.
  Tesseroids: forward modeling gravitational fields in spherical coordinates,
  \emph{Geophysics},
  \DOI{10.1190/geo2015-0204.1}.
  \newline
  \GitHub{pinga-lab/paper-tesseroids}
  \\
  ~ &
  \Dio, \Me, \Val.
  How two gravity-gradient inversion methods can be used to reveal different
  geologic features of ore deposit - A case study from the Quadrilátero
  Ferrífero (Brazil),
  \emph{Journal of Applied Geophysics},
  \DOI{10.1016/j.jappgeo.2016.04.011}.
  \\
\Year{2015}  &
  \Bi, \Dai, \Val, \Me.
  Estimation of the total magnetization direction of approximately spherical
  bodies,
  \emph{Nonlinear Processes in Geophysics},
  \DOI{10.5194/npg-22-215-2015}.
  \OA
  \newline
  \GitHub{pinga-lab/Total-magnetization-of-spherical-bodies}
  \\
\Year{2014}  &
  \Dio, \Me, \Val.
  Imaging iron ore from the Quadrilátero Ferrífero (Brazil) using geophysical
  inversion and drill hole data,
  \emph{Ore Geology Reviews},
  \DOI{10.1016/j.oregeorev.2014.02.011}.
  \\
\Year{2013}  &
  \Figura, \Val, \Me, \Bi, \JB.
  Estimating the nature and the horizontal and vertical positions of 3D
  magnetic sources using Euler deconvolution,
  \emph{Geophysics},
  \DOI{10.1190/geo2012-0515.1}.
  \\
  ~ &
  \Bi, \Val, \Me.
  Polynomial equivalent layer,
  \emph{Geophysics},
  \DOI{10.1190/geo2012-0196.1}.
  \\
\Year{2012}  &
  \Me, \Val.
  Robust 3D gravity gradient inversion by planting anomalous densities,
  \emph{Geophysics},
  \DOI{10.1190/geo2011-0388.1}.
  \newline
  \GitHub{pinga-lab/paper-planting-densities}
\end{EntriesTable}


\subsection{Peer-Reviewed Conference proceedings}

\begin{EntriesTable}
\Year{2014}  &
  \Figura, \Val, \Me, \Bi, \JB.
  A Single Euler Solution Per Anomaly,
  \emph{76th EAGE Conference and Exhibition 2014},
  \DOI{10.3997/2214-4609.20140891}.
  \\
\Year{2013}  &
  \Me, \Bi, \Val.
  Modeling the Earth with Fatiando a Terra,
  \emph{Proceedings of the 12th Python in Science Conference}.
  \DOI{10.25080/Majora-8b375195-010}.
  \\
\Year{2012}  &
  \Me, \Val.
  Use of the ``shape-of-anomaly'' data misfit in 3D inversion by planting
  anomalous densities,
  \emph{SEG Technical Program Expanded Abstracts},
  \DOI{10.1190/segam2012-0383.1}.
  \\
  ~ &
  \Dio, \Me, \YLi, \Val, \BragaVale, \Angeli, \Peres.
  Iron ore interpretation using gravity-gradient inversions in the Carajás, Brazil.
  \emph{SEG Technical Program Expanded Abstracts},
  \DOI{10.1190/segam2012-0525.1}.
  \\
\Year{2011}  &
  \Me, \Everton, \Carla, \Eder.
  Optimal forward calculation method of the Marussi tensor due to a geologic
  structure at GOCE height,
  \emph{Proceedings of the 4th International GOCE User Workshop}.
  \DOI{10.6084/m9.figshare.92624}.
  \\
  ~ &
  \Me, \Val.
  Robust 3D gravity gradient inversion by planting anomalous densities,
  \emph{SEG Technical Program Expanded Abstracts},
  \DOI{10.1190/1.3628201}.
  \\
  ~ &
  \Me, \Val.
  3D gravity inversion by planting anomalous densities.
  \emph{12th International Congress of the Brazilian Geophysical Society},
  \DOI{10.1190/sbgf2011-179}.
  \\
  ~ &
  \Me, \Val.
  3D gravity gradient inversion by planting density anomalies.
  \emph{73th EAGE Conference and Exhibition incorporating SPE EUROPEC},
  \DOI{10.3997/2214-4609.20149567}.
  \\
  ~ &
  \Dio, \Me, \Val, \BragaVale, \Gomes.
  In-depth imaging of an iron orebody from Quadrilatero Ferrifero using 3D
  gravity gradient inversion,
  \emph{SEG Technical Program Expanded Abstracts},
  \DOI{10.1190/1.3628219}.
  \\
  ~ &
  \Dio, \Val, \Me, \BragaVale.
  Inversão de Dados de Aerogradiometria Gravimétrica 3D-FTG Aplicada a
  Exploração Mineral na Região do Quadrilátero Ferrífero,
  \emph{12th International Congress of the Brazilian Geophysical Society},
  \DOI{10.1190/sbgf2011-243}.
\end{EntriesTable}

\subsection{Open Datasets}

\begin{EntriesTable}
\Year{2017}  &
  \Me, \Val.
  A gravity-derived Moho model for South America: source code, data, and
  model results from ``Fast non-linear gravity inversion in spherical
  coordinates with application to the South American Moho''.
  \DOI{10.6084/m9.figshare.3987267}
\end{EntriesTable}


%%%%%%%%%%%%%%%%%%%%%%%%%%%%%%%%%%%%%%%%%%%%%%%%%%%%%%%%%%%%%%%%%%%%%%%%%%%%%%%
\section{Open-source Software}

\begin{EntriesTable}
  \Duration{2017}{\Ongoing} &
  \textbf{PyGMT}
  \newline
  A Python interface for the Generic Mapping Tools
  \newline
  Role: \textit{Creator and core developer}
  \newline
  \GitHub{GenericMappingTools/pygmt}
  \newline
  Website: \href{https://www.pygmt.org}{www.pygmt.org}
  \\
  \Duration{2017}{\Ongoing} &
  \textbf{The Generic Mapping Tools (GMT)}
  \newline
  A data processing and mapping toolbox for the Earth, Ocean, and Planetary Science
  \newline
  Role: \textit{Core team and community management}
  \newline
  \GitHub{GenericMappingTools/gmt}
  \newline
  Website: \href{https://www.generic-mapping-tools.org}{www.generic-mapping-tools.org}
  \\
  \Duration{2010}{\Ongoing} &
  \textbf{Fatiando a Terra}
  \newline
  Python tools for geophysical data processing, forward modeling, and inversion
  \newline
  Role: \textit{Creator, main developer, project leadership}
  \newline
  \GitHub{fatiando}
  \newline
  Website: \href{https://www.fatiando.org}{www.fatiando.org}
  \\
  \Duration{2009}{2016} &
  \textbf{Tesseroids}
  \newline
  Forward modeling of gravitational fields in spherical coordinates
  \newline
  Role: \textit{Creator and sole developer}
  \newline
  \GitHub{leouieda/tesseroids}
  \newline
  Website: \href{http://www.tesseroids.org}{www.tesseroids.org}
\end{EntriesTable}


%%%%%%%%%%%%%%%%%%%%%%%%%%%%%%%%%%%%%%%%%%%%%%%%%%%%%%%%%%%%%%%%%%%%%%%%%%%%%%%
\section{Teaching}

\subsection{Undergraduate}

\begin{EntriesTable}
  \Duration{2020}{\Ongoing}  &
  ENVS101/106: Study Skills and GIS (tutorial)
  \newline
  Leading small group tutorials and a Python programming workshop
  \newline
  \textit{\LIV}
  \\
  \Duration{2020}{\Ongoing}  &
  ENVS398: Global Geophysics and Geodynamics
  \newline
  Teaching lithosphere dynamics (50\% of module)
  \newline
  Module coordinator from 2021
  \newline
  \textit{\LIV}
  \\
  \Duration{2020}{\Ongoing}  &
  ENVS258: Environmental Geophysics
  \newline
  Teaching remote sensing, gravimetry, and Python programming
  ($\sim$50\% of module)
  \newline
  \textit{\LIV}
  \\
  \Duration{2019}{\Ongoing}  &
  ENVS363: Geophysical Exploration Techniques (field)
  \newline
  Part of the teaching team for geophysical field methods
  \newline
  \textit{\LIV}
  \\
  \Duration{2019}{\Ongoing}  &
  ENVS123: Introduction to Geoscience and Earth History
  \newline
  Lectures on: Earth's internal structure; gravity and isostasy
  \newline
  \textit{\LIV}
  \\
  \Duration{2014}{2016}  &
  Special Mathematics I: Introduction to Programming and Numerical Analysis
  \newline
  \textit{\UERJ}
  \newline
  \GitHub{mat-esp/about}
  \\
  \Duration{2014}{2016}  &
  Geophysics I: Gravity and magnetic methods
  \newline
  \textit{\UERJ}
  \newline
  \GitHub{leouieda/geofisica1}
  \\
  \Duration{2014}{2016}  &
  Geophysics II: Exploration Seismology
  \newline
  \textit{\UERJ}
  \newline
  \GitHub{leouieda/geofisica2}
  \\
  \Year{2015}  &
  Introduction to Geology
  \newline
  \textit{\UERJ}
\end{EntriesTable}


\subsection{Workshops \& Short Courses}

\begin{EntriesTable}
%\Year{future}  &
\Year{2020} &
  Let's build a geophysical inversion with Python
  \newline
  \textit{IRTG-2379 Graduate School: Modern Inverse Problems}
  \newline
  \textit{RWTH Aachen University} (online)
  \newline
  \GitHub{compgeolab/2020-aachen-inverse-problems}
  \\
  ~ &
  The Generic Mapping Tools for Geodesy
  \newline
  \textit{UNAVCO} (online)
  \newline
  \GitHub{GenericMappingTools/2020-unavco-course}
  \\
  ~  &
  From scattered data to gridded products using Verde
  \newline
  \textit{Transform 2020} (online)
  \newline
  \Youtube{-xZdNdvzm3E}
  \newline
  \GitHub{fatiando/transform2020}
  \\
\Year{2019}  &
  Best Practices for Developing and Sustaining Your Open-Source Research Software
  \newline
  \textit{AGU Fall Meeting 2019}
  \newline
  \GitHub{agu-ossi/2019-agu-oss}
  \\
  ~  &
  Become a Generic Mapping Tools Contributor Even If You Can't Code
  \newline
  \textit{AGU Fall Meeting 2019}
  \\
  ~  &
  The Generic Mapping Tools for Geodesy
  \newline
  \textit{Scripps Institution of Oceanography} and \textit{UNAVCO}
  \newline
  \GitHub{GenericMappingTools/2019-unavco-course}
  \\
  ~  &
  Introduction to Python Workshop (Earth Sciences REU program)
  \newline
  \textit{Department of Geology and Geophysics, \UHM}
  \newline
  \GitHub{leouieda/2019-06-reu-python}
  \\
\Year{2018}  &
  Best Practices for Modern Open-Source Research Codes
  \newline
  \textit{AGU Fall Meeting 2018}
  \newline
  \GitHub{agu-ossi/2018-agu-oss}
  \\
  ~  &
  Git and Github: What are their uses? Are they worth the effort? Let's find out!
  \newline
  \textit{ASPRS UHM Student Chapter, \UHM}
  \\
\Year{2017}  &
  Introduction to Python
  \newline
  \textit{Department of Geology and Geophysics, \UHM}
  \newline
  \GitHub{leouieda/python-hawaii-2017}
  \\
\Year{2016}  &
  Python for Geologists (SAGEO)
  \newline
  \textit{Faculdade de Geologia, \UERJ}
  \newline
  \GitHub{leouieda/python-geologia-2016}
  \\
  ~  &
  Python for Earth Scientists (IAG Summer School)
  \newline
  \textit{Departamento de Geofísica, Universidade de São Paulo}
  \newline
  \GitHub{leouieda/verao2016}
  \\
\Year{2014}  &
  Introduction to Geophysical Inversion
  \newline
  \textit{Instituto de Geociências, Universidade de Brasília}
  \newline
  \GitHub{pinga-lab/inversao-unb-2014}
  \\
\Year{2011}  &
  Introduction to Geophysical Inversion (IAG Summer School)
  \newline
  \textit{Departamento de Geofísica, Universidade de São Paulo}
  \newline
  \GitHub{pinga-lab/inversao-iag-2012}
\end{EntriesTable}


%%%%%%%%%%%%%%%%%%%%%%%%%%%%%%%%%%%%%%%%%%%%%%%%%%%%%%%%%%%%%%%%%%%%%%%%%%%%%%%
\section{Student supervision}

\subsection{P\lowercase{h}D}

\begin{EntriesTable}
\Duration{2017}{\Ongoing}  &
  Santiago R. Soler (co-Advising)
  \newline
  Universidad Nacional de San Juan, Argentina.
  \newline
  Advisor: Mario E. Gimenez
\end{EntriesTable}

\subsection{Master's}

\begin{EntriesTable}
\Duration{2020}{2021}  &
  Aidan Hernaman
  \newline
  \LIV, UK.
\end{EntriesTable}

\subsection{Undergraduate}

\begin{EntriesTable}
\Duration{2020}{2021}  &
  Majed M.A. Abura, Ali A.A. Alhazmi, Daniel P. Gilbert, and Mustafa M.M.
  Alordowny
  \newline
  \LIV, UK.
  \\
\Duration{2019}{2020}  &
  Lottie Cooper, Steven Heer, Charles Thomson, and Alexander Borges
  \newline
  \LIV, UK.
  \\
\Duration{2015}{2017}  &
  Vinicius V. Riguete
  \newline
  \UERJ, Brazil.
  \\
\end{EntriesTable}


%%%%%%%%%%%%%%%%%%%%%%%%%%%%%%%%%%%%%%%%%%%%%%%%%%%%%%%%%%%%%%%%%%%%%%%%%%%%%%%
\section{Outreach}

I maintain a blog about my research, geoscience, and programming at
\href{https://www.leouieda.com/blog}{leouieda.com/blog}
\\

\begin{EntriesTable}
\Year{2018}  &
  Interviewed by the geoscience podcast \textit{Don't Panic Geocast}, episode 166
  \textit{``You are headed to a warm and sunny place''}:
  \href{http://www.dontpanicgeocast.com/?p=638}{dontpanicgeocast.com/?p=638}
  \\
\Year{2017}  &
  Volunteer for the \textit{Hour of Code} at Salt Lake Elementary School, Honolulu,
  USA.
  \\
\Year{2016}  &
  Interviewed by the geoscience podcast \textit{Undersampled Radio}, episode
  \textit{``Open Sourcery''}:
  \href{https://undersampledrad.io/home/2016/7/open-sourcery}{undersampledrad.io/home/2016/7/open-sourcery}
\end{EntriesTable}

Geophysical tutorials for the SEG publication \textit{The Leading Edge}:
\\

\begin{EntriesTable}
\Year{2017}  &
  \Me.
  Step-by-step NMO correction,
  \emph{The Leading Edge},
  \DOI{10.1190/tle36020179.1}.
  \OA
  \\
\Year{2014}  &
  \Me, \Bi, \Val.
  Geophysical tutorial: Euler deconvolution of potential-field data,
  \emph{The Leading Edge},
  \DOI{10.1190/tle33040448.1}.
  \OA
\end{EntriesTable}


%%%%%%%%%%%%%%%%%%%%%%%%%%%%%%%%%%%%%%%%%%%%%%%%%%%%%%%%%%%%%%%%%%%%%%%%%%%%%%%
\section{Presentations}

\begin{EntriesTable}
%\Year{\Future}  &
  %\Invited{}
  %\Me.
  %Geophysical research powered by open-source,
  %\emph{Christian-Albrechts-Universität zu Kiel},
  %Kiel, Germany.
  %\\
\Year{2020}  &
  \Invited{}
  \Me.
  Geophysical research powered by open-source,
  \emph{Christian-Albrechts-Universität zu Kiel},
  Kiel, Germany.
  \\
  ~ &
  \Invited{}
  \Me.
  Geophysical research powered by open-source,
  \emph{Departamento de Geofísica, IAG, Universidade de São Paulo},
  São Paulo, Brazil.
  \href{https://www.leouieda.com/2020-06-18-usp}{leouieda.com/2020-06-18-usp}.
  \Youtube{VqI8BX1Yg54}.
  \\
  ~ &
  \Invited{}
  \Me.
  Geophysical research powered by open-source,
  \emph{Technische Universität Bergakademie Freiberg},
  Freiberg, Germany.
  \href{https://www.leouieda.com/2020-06-04-freiberg}{leouieda.com/2020-06-04-freiberg}.
  \\
  ~ &
  \Me, \Santiago.
  Evaluating the accuracy of equivalent-source predictions using
  cross-validation,
  \emph{EGU 2020},
  Vienna, Austria.
  \DOI{10.5194/egusphere-egu2020-15729}.
  \DOI{10.6084/m9.figshare.12245372}.
  \\
  ~ &
  \Invited{}
  \Me.
  Geophysical research powered by open-source,
  \emph{Geographic Data Science Lab, University of Liverpool},
  Liverpool, UK.
  \href{https://www.leouieda.com/liverpool-gdsl-2020}{leouieda.com/liverpool-gdsl-2020}.
  \\
\Year{2019}  &
  \Me, \Paul.
  PyGMT: Accessing the Generic Mapping Tools from Python,
  \emph{AGU 2019},
  San Francisco, USA.
  \DOI{10.6084/m9.figshare.11320280}.
  \\
  ~ &
  \Me.
  Building the foundations for open-source geophysics,
  \emph{\LIVEARTH, \LIV},
  UK.
  \DOI{10.6084/m9.figshare.10255832}.
  \\
\Year{2018}  &
  \Me, \Eric, \Paul, \David.
  Coupled Interpolation of Three-component GPS Velocities,
  \emph{AGU 2018},
  Washington DC, USA.
  \DOI{10.6084/m9.figshare.7440683}.
  \\
  ~ &
  \Me.
  Machine Learning Lessons for Geophysics,
  \emph{Department of Earth Sciences, \UHM},
  Honolulu, USA.
  \DOI{10.6084/m9.figshare.7203344}.
  \\
  ~ &
  \Me, \Paul.
  Building an object-oriented Python interface for the Generic Mapping Tools,
  \emph{Scipy 2018},
  Austin, USA.
  \DOI{10.6084/m9.figshare.6814052}.
  \Youtube{6wMtfZXfTRM}
  \\
  ~ &
  \Me, \David, \Paul.
  Joint Interpolation of 3-component GPS Velocities Constrained by
  Elasticity,
  \emph{AOGS $15^{th}$ Annual Meeting},
  Honolulu, USA.
  \DOI{10.6084/m9.figshare.6387467}.
  \\
  ~ &
  \Me, \Paul.
  Integrating the Generic Mapping Tools with the Scientific Python Ecosystem,
  \emph{AOGS $15^{th}$ Annual Meeting},
  Honolulu, USA.
  \DOI{10.6084/m9.figshare.6399944}.
  \\
\Year{2017}  &
  \Invited{}
  \Me, \Paul.
  Nurturing reliable and robust open-source scientific software,
  \emph{AGU Fall Meeting 2017},
  New Orleans, USA.
  \Youtube{0GO4ZZ5Ry6M}
  \\
  ~  &
  \Me, \Paul.
  A modern Python interface for the Generic Mapping Tools,
  \emph{AGU Fall Meeting 2017},
  New Orleans, USA.
  \DOI{10.6084/m9.figshare.5662411}.
  \\
  ~  &
  \Me, \Paul.
  Bringing the Generic Mapping Tools to Python,
  \emph{Scipy 2017},
  Austin, USA.
  \DOI{10.6084/m9.figshare.7635833}.
  \Youtube{93M4How7R24}
  \\
  ~ &
  \Me.
  Inverting gravity to map the Moho: A new method and the open source
  software that made it possible,
  \emph{Department of Geology and Geophysics, \UHM},
  Honolulu, USA.
  \DOI{10.6084/m9.figshare.4779766}.
  \\
\Year{2016}  &
  \Invited{}
  \Me.
  Fatiando a Terra: construindo uma base para ensino e pesquisa de geofísica,
  \emph{Observatório Nacional},
  Rio de Janeiro, Brazil.
  \\
\Year{2015}  &
  \Invited{}
  \Me.
  Fatiando a Terra: construindo uma base para ensino e pesquisa de geofísica,
  \emph{Universidade de São Paulo},
  São Paulo, Brazil.
  \DOI{10.6084/m9.figshare.1381870}.
  \\
\Year{2014}  &
  \Me, \Bi, \Val.
  Using Fatiando a Terra to solve inverse problems in geophysics,
  \emph{Scipy 2014},
  Austin, USA.
  \DOI{10.6084/m9.figshare.1089987}.
  \\
  ~ &
  \Me, \Val.
  Gravity inversion in spherical coordinates using tesseroids,
  \emph{EGU General Assembly 2014},
  Vienna, Austria.
  \DOI{10.6084/m9.figshare.1155457}.
  \\
\Year{2013}  &
  \Me, \Bi, \Val.
  Modeling the Earth with Fatiando a Terra,
  \emph{Scipy 2013},
  Austin, USA.
  \DOI{10.25080/Majora-8b375195-010}.
  \Youtube{Ec38h1oB8cc}
  \\
  ~ &
  \Me, \Val.
  3D magnetic inversion by planting anomalous densities,
  \emph{AGU Meeting of the Americas},
  Cancun, Mexico.
  \DOI{10.6084/m9.figshare.703651}.
  \\
\Year{2012}  &
  \Dio, \Me, \YLi, \Val, \BragaVale, \Angeli, \Peres.
  Iron ore interpretation using gravity-gradient inversions in the Carajás,
  Brazil,
  \emph{SEG Annual Meeting 2012},
  Las Vegas, USA.
  \DOI{10.6084/m9.figshare.156865}.
  \\
  ~ &
  \Me, \Val.
  Use of the ``shape-of-anomaly'' data misfit in 3D inversion by planting
  anomalous densities,
  \emph{SEG Annual Meeting 2012},
  Las Vegas, USA.
  \DOI{10.6084/m9.figshare.156864}.
  \\
  ~ &
  \Me, \Val.
  Rapid 3D inversion of gravity and gravity gradient data to test geologic
  hypotheses,
  \emph{International Symposium on Gravity, Geoid and Height Systems},
  Venice, Italy.
  \DOI{10.6084/m9.figshare.156859}.
  \\
\Year{2011}  &
  \Me, \Val.
  Robust 3D gravity gradient inversion by planting anomalous densities,
  \emph{SEG Annual Meeting 2011},
  San Antonio, USA.
  \DOI{10.6084/m9.figshare.156863}.
  \\
  ~ &
  \Me, \Val.
  3D gravity inversion by planting anomalous densities,
  \emph{Internation Congress of the Brazilian Geophysical Society},
  Rio de Janeiro, Brazil.
  \DOI{10.6084/m9.figshare.156861}.
  \\
  ~ &
  \Me, \Everton, \Carla, \Eder.
  Optimal forward calculation method of the Marussi tensor due to a geologic
  structure at GOCE height,
  \emph{4th International GOCE User Workshop},
  Munich, Germany.
  \DOI{10.6084/m9.figshare.92624}.
  \\
  ~ &
  \Me, \Val.
  3D gravity gradient inversion by planting density anomalies,
  \emph{73th EAGE Conference and Exhibition incorporating SPE EUROPEC},
  Vienna, Austria.
  \DOI{10.6084/m9.figshare.91511}.
  \\
\Year{2010}  &
  \Me, \Naomi, \Carla.
  Computation of the gravity gradient tensor due to topographic masses using
  tesseroids,
  \emph{AGU Meeting of the Americas},
  Foz do Iguaçu, Brazil.
  \DOI{10.6084/m9.figshare.156858}.
  \\
\Year{2008}  &
  \Me, \Naomi.
  Utilização de tesseróides na modelagem de dados de gradiometria
  gravimétrica,
  \emph{XIII Simpósio de Iniciação Científica do IAG-USP},
  São Paulo, Brazil.
  \DOI{10.6084/m9.figshare.4779760}.
  \\
\Year{2006}  &
  \Me, \Manoel.
  Paleomagnetismo e mineralogia magnética dos diques cambrianos de Maravilhas
  e Prata (PB),
  \emph{XI Simpósio de Iniciação Científica do IAG/USP},
  São Paulo, Brazil.
  \DOI{10.6084/m9.figshare.4779769}.
\end{EntriesTable}


%%%%%%%%%%%%%%%%%%%%%%%%%%%%%%%%%%%%%%%%%%%%%%%%%%%%%%%%%%%%%%%%%%%%%%%%%%%%%%%
\section{Academic Service \& Affiliations}

\subsection{Editor}

\begin{EntriesTable}
  \Duration{2019}{\Ongoing} & Topic editor for the \textit{Journal of Open Source Software}
\end{EntriesTable}

\subsection{Reviewer}

Geophysical Journal International
--
Journal of Geodesy
--
Pure and Applied Geophysics
--
Journal of Applied Geophysics
--
Geophysical Prospecting
--
Geophysics
--
Central European Journal of Geosciences
--
Computers \& Geosciences
--
Journal of Open Source Software

\subsection{Committees}

\begin{EntriesTable}
\Duration{2020}{\Ongoing} &
  Department committee for web presence (website, social media, etc),
  \LIV.
  \\
\Duration{2020}{\Ongoing} &
  Earth Sciences representative at the Early Career Academic (ECA) forum,
  \LIV.
  \\
\Year{2015} &
  Chairman of the Election Committee for the deans of the University and the School of
  Geology, \UERJ.
  \\
\Duration{2015}{2017} &
  Faculty Advisor for the Student Chapter of the Socienty of Exploration Geophysicists
  (SEG) at the \UERJ.
\end{EntriesTable}

\subsection{Convener}

\begin{EntriesTable}
%\Year{future} &
\Year{2021} &
  Session: The evolving open-science landscape in geosciences.
  \newline
  \JFarquharson, \AKushnir, \Me, \FWadsworth.
  \newline
  \emph{EGU 2021}, Vienna, Austria.
  \\
  ~ &
  Session: Acquisition and processing of gravity and magnetic field data and their
  integrative interpretation.
  \newline
  \JEbbing, \Carla, \AGuy, \MKaban, \Me.
  \newline
  \emph{EGU 2021}, Vienna, Austria.
  \\
\Year{2019} &
  Townhall: Update and Future Directions of the Open-Source Software Initiative.
  \newline
  \Me, \Lindsey, \Lion, \Rene, \Bane.
  \newline
  \emph{AGU 2019}, San Francisco, USA.
  \\
  ~ &
  Session: NS21A - A Tour of Open-Source Software Packages for the Geosciences.
  \newline
  \Lindsey, \Rene, \Me, \Jens.
  \newline
  \emph{AGU 2019}, San Francisco, USA.
  \\
\Year{2018} &
  Townhall: The role of an open-source software initiative within the AGU.
  \newline
  \Lindsey, \Lion, \Me.
  \newline
  \emph{AGU 2018}, Washington DC, USA.
\end{EntriesTable}

\subsection{Affiliations}

\begin{EntriesTable}
  \Duration{2020}{\Ongoing} & Royal Astronomical Society
  \\
  \Duration{2014}{\Ongoing} & \href{https://softwareunderground.org}{Software Underground}
  \\
  \Duration{2014}{\Ongoing} & European Geosciences Union
  \\
  \Duration{2010}{\Ongoing} & American Geophysical Union
  \\
  \Duration{2011}{2019} & Society of Exploration Geophysicists
\end{EntriesTable}

\subsection{Other}

\begin{EntriesTable}
  \Duration{2019}{2020} & Member of the EarthArXiv Advisory Council (2 year tenure)
\end{EntriesTable}


%%%%%%%%%%%%%%%%%%%%%%%%%%%%%%%%%%%%%%%%%%%%%%%%%%%%%%%%%%%%%%%%%%%%%%%%%%%%%%%
\section{Languages}

\TablePad
\begin{tabularx}{\textwidth}{@{}p{0.15\textwidth} p{0.85\textwidth}@{}}
  Portuguese & Native
  \\
  English & IELTS: CEFR Level C2 (mastery or proficiency) obtained in 2019
\end{tabularx}

\end{document}
